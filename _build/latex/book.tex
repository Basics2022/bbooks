%% Generated by Sphinx.
\def\sphinxdocclass{jupyterBook}
\documentclass[letterpaper,10pt,italian]{jupyterBook}
\ifdefined\pdfpxdimen
   \let\sphinxpxdimen\pdfpxdimen\else\newdimen\sphinxpxdimen
\fi \sphinxpxdimen=.75bp\relax
\ifdefined\pdfimageresolution
    \pdfimageresolution= \numexpr \dimexpr1in\relax/\sphinxpxdimen\relax
\fi
%% let collapsible pdf bookmarks panel have high depth per default
\PassOptionsToPackage{bookmarksdepth=5}{hyperref}
%% turn off hyperref patch of \index as sphinx.xdy xindy module takes care of
%% suitable \hyperpage mark-up, working around hyperref-xindy incompatibility
\PassOptionsToPackage{hyperindex=false}{hyperref}
%% memoir class requires extra handling
\makeatletter\@ifclassloaded{memoir}
{\ifdefined\memhyperindexfalse\memhyperindexfalse\fi}{}\makeatother

\PassOptionsToPackage{warn}{textcomp}

\catcode`^^^^00a0\active\protected\def^^^^00a0{\leavevmode\nobreak\ }
\usepackage{cmap}
\usepackage{fontspec}
\defaultfontfeatures[\rmfamily,\sffamily,\ttfamily]{}
\usepackage{amsmath,amssymb,amstext}
\usepackage{polyglossia}
\setmainlanguage{italian}



\setmainfont{FreeSerif}[
  Extension      = .otf,
  UprightFont    = *,
  ItalicFont     = *Italic,
  BoldFont       = *Bold,
  BoldItalicFont = *BoldItalic
]
\setsansfont{FreeSans}[
  Extension      = .otf,
  UprightFont    = *,
  ItalicFont     = *Oblique,
  BoldFont       = *Bold,
  BoldItalicFont = *BoldOblique,
]
\setmonofont{FreeMono}[
  Extension      = .otf,
  UprightFont    = *,
  ItalicFont     = *Oblique,
  BoldFont       = *Bold,
  BoldItalicFont = *BoldOblique,
]



\usepackage[Sonny]{fncychap}
\ChNameVar{\Large\normalfont\sffamily}
\ChTitleVar{\Large\normalfont\sffamily}
\usepackage[,numfigreset=1,mathnumfig]{sphinx}

\fvset{fontsize=\small}
\usepackage{geometry}


% Include hyperref last.
\usepackage{hyperref}
% Fix anchor placement for figures with captions.
\usepackage{hypcap}% it must be loaded after hyperref.
% Set up styles of URL: it should be placed after hyperref.
\urlstyle{same}


\usepackage{sphinxmessages}



        % Start of preamble defined in sphinx-jupyterbook-latex %
         \usepackage[Latin,Greek]{ucharclasses}
        \usepackage{unicode-math}
        % fixing title of the toc
        \addto\captionsenglish{\renewcommand{\contentsname}{Contents}}
        \hypersetup{
            pdfencoding=auto,
            psdextra
        }
        % End of preamble defined in sphinx-jupyterbook-latex %
        

\title{basics-books}
\date{28 gen 2025}
\release{}
\author{basics}
\newcommand{\sphinxlogo}{\vbox{}}
\renewcommand{\releasename}{}
\makeindex
\begin{document}

\pagestyle{empty}
\sphinxmaketitle
\pagestyle{plain}
\sphinxtableofcontents
\pagestyle{normal}
\phantomsection\label{\detokenize{intro::doc}}


\sphinxAtStartPar
This is the main landing page of the \sphinxstylestrong{basics} project.

\sphinxAtStartPar
The project will be first developed in Italian, at least for material targeting high\sphinxhyphen{}schools. Please, rely on the translator of your browser, or have a little patience waiting for translation, or \sphinxhyphen{} even better \sphinxhyphen{} feel free to contribute through \sphinxhref{https://github.com/Basics2022}{Github repos}.

\sphinxAtStartPar
Material for university is likely to be a jupyter\sphinxhyphen{}book version of some hand\sphinxhyphen{}written notes already available at \sphinxurl{https://basics.altervista.org}.
\subsubsection*{High\sphinxhyphen{}school}



\sphinxAtStartPar
\sphinxhref{https://basics2022.github.io/bbooks-programming-hs}{\sphinxstylestrong{Introduction to programming}} \sphinxstylestrong{work\sphinxhyphen{}in\sphinxhyphen{}progress}

\sphinxAtStartPar
\sphinxhref{https://basics2022.github.io/bbooks-math-miscellanea-hs}{\sphinxstylestrong{Mathematics}} \sphinxstylestrong{work\sphinxhyphen{}in\sphinxhyphen{}progress}

\sphinxAtStartPar
\sphinxhref{https://basics2022.github.io/bbooks-physics-hs}{\sphinxstylestrong{Physics}} \sphinxstylestrong{work\sphinxhyphen{}in\sphinxhyphen{}progress}

\sphinxAtStartPar
\sphinxhref{https://basics2022.github.io/bbooks-chemistry-hs}{\sphinxstylestrong{Chemistry}} \sphinxstylestrong{work\sphinxhyphen{}in\sphinxhyphen{}progress}

\sphinxAtStartPar
\sphinxhref{https://basics2022.github.io/bbooks-economics-hs}{\sphinxstylestrong{Economics}} \sphinxstylestrong{todo}

\sphinxAtStartPar
\sphinxhref{https://basics2022.github.io/bbooks-philosophy}{\sphinxstylestrong{Philosopy}} \sphinxstylestrong{todo}


\subsubsection*{University and more}



\sphinxAtStartPar
\sphinxstylestrong{Mathematics}
\begin{itemize}
\item {} 
\sphinxAtStartPar
\sphinxhref{https://basics2022.github.io/bbooks-math-miscellanea}{\sphinxstylestrong{Mathematics}} \sphinxstylestrong{work\sphinxhyphen{}in\sphinxhyphen{}progress}

\end{itemize}

\sphinxAtStartPar
\sphinxstylestrong{Physics}
\begin{itemize}
\item {} 
\sphinxAtStartPar
\sphinxhref{https://basics2022.github.io/bbooks-physics-mechanics}{\sphinxstylestrong{Classical mechanics}} \sphinxstylestrong{work\sphinxhyphen{}in\sphinxhyphen{}progress}

\item {} 
\sphinxAtStartPar
\sphinxhref{https://basics2022.github.io/bbooks-physics-thermodynamics}{\sphinxstylestrong{Classical thermodynamics}} \sphinxstylestrong{work\sphinxhyphen{}in\sphinxhyphen{}progress}

\item {} 
\sphinxAtStartPar
\sphinxhref{https://basics2022.github.io/bbooks-physics-electromagnetism}{\sphinxstylestrong{Classical electromagnetism}} \sphinxstylestrong{work\sphinxhyphen{}in\sphinxhyphen{}progress}

\item {} 
\sphinxAtStartPar
\sphinxhref{https://basics2022.github.io/bbooks-physics-continuum-mechanics}{\sphinxstylestrong{Continuum mechanics}}: introduction to continuum mechanics, solid mechanics, and fluid mechanics \sphinxstylestrong{work\sphinxhyphen{}in\sphinxhyphen{}progress}

\item {} 
\sphinxAtStartPar
\sphinxhref{https://basics2022.github.io/bbooks-physics-modern}{\sphinxstylestrong{Modern physics}} \sphinxstylestrong{work\sphinxhyphen{}in\sphinxhyphen{}progress}

\end{itemize}
\subsubsection*{Insights}

\sphinxstepscope


\chapter{About}
\label{\detokenize{ch/about:about}}\label{\detokenize{ch/about:bbooks-about}}\label{\detokenize{ch/about::doc}}

\section{Credits and License}
\label{\detokenize{ch/about:credits-and-license}}
\sphinxAtStartPar
…\sphinxstylestrong{todo}…


\section{Timeline}
\label{\detokenize{ch/about:timeline}}
\sphinxAtStartPar
\sphinxstylestrong{2024 Oct 10 \sphinxhyphen{} Initial commit of the bbooks\sphinxhyphen{}template.}

\sphinxAtStartPar
\sphinxstylestrong{2024 Nov 13 \sphinxhyphen{} Jupyter books for interactive open textbooks at TUDelft.}
On the 2024 Nov 13, I’ve found out that some people working at TUDelft are using the same approach for building an open textbook project, hosted on \sphinxstylestrong{Gitlab}, collecting books for university. I’m sure I’ll find some useful tips for this project, either in their manual or \sphinxhref{https://interactivetextbooks.tudelft.nl/open-textbooks-demonstration/index.html}{demonstration with Jupyter books}, or browsing in their \sphinxhref{https://gitlab.tudelft.nl/opentextbooks}{repo}.
Maybe I’ll contact them in the future, as soon as I have an English version of these documents.
\begin{itemize}
\item {} 
\sphinxAtStartPar
Try to understand if there’s a good way to include \sphinxstylestrong{exercises}, without the need for repeating code with only\sphinxhyphen{}latex and only\sphinxhyphen{}html blocks. See at the end \sphinxhref{https://gitlab.tudelft.nl/opentextbooks/open-textbooks-demonstration/-/tree/master?ref\_type=heads}{repository page}

\item {} 
\sphinxAtStartPar
Should I move the project from Github to Gitlab? Why did I choose Github over Gitlab? I suspect there was some good reason to make this choice \sphinxhyphen{} that I’d have never done otherwise \sphinxhyphen{} but I can’t remember why…

\end{itemize}







\renewcommand{\indexname}{Indice}
\printindex
\end{document}